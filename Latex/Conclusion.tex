\chapter{Conclusion and Future Work}
\section{Conclusion}
The recently introduced Delay-Tolerant Network architecture assembles networks with different latencies, bandwith limitations or node longevities together and makes the interoperation between them achievable. Although DTN's high level security issues have been analysed and implemented in many publications, there is no a proper security protocol specifically focused on its courier-dependent communication network. Two of the related security protocols - Bundle Security Protocol and DTN Anonymity and Secure Architecture, have been examined and discussed, consequently, it has been shown that there are still some improvements to be made.

Then a new security protocol specifically tailored for courier-dependent communication is created, named Courier-Dependent Security Protocol (CDSP). It has following main security achievements - authentication of originator and recipient, authenticity of origin of the message, confidentiality of the message content and deniability for originator and recipient, and the time efficiency of the protocol is raised to a high priority. To show the protocol, firstly the whole system that the protocol will be working in is fully described. Then is the protocol specification, specifies the detailed procedure of running CDSP step by step. The last part of the protocol illustration summaries all security properties of CDSP, showing its capability and then proves every one of those properties. 

After CDSP has been thoroughly described, a concrete implementation of this protocol is developed in JAVA. The program mainly consists of a core library which provides all essential functions to run CDSP and a runnable application uses the core library. The core library is designed with emphasized extensibility and reusability, thus it can be easily modified and extended after the prototype released. The program architecture is illustrated by understandable figures and its security-related designs are fully documented afterwards.

Finally the application is tested for performance evaluation. Two main aspects - transferring latency and scalability have been taken into account, the capability of the application is shown and some interesting conclusions about the protocol behaviour have been drawn after the evaluation.

\section{Discussion and Future Work}
The most obvious defect of this protocol system is that the success delivery of messages is never guaranteed by courier. Due to the efficiency and simplicity concerns, couriers are never required to authenticate to message creator, which means any entity can claim itself as courier and get the data. It leads to the problem that message creator will never know whether her message has been submitted to the targeted courier, neither will her know whether her message will eventually be delivered to the recipient. This kind of uncertainty can be problematic in some scenarios where the success delivery guarantee is strongly demanded - like transmitting military intelligence. One potential solution can reduce the extend of its uncertainty by forcing couriers to authenticated to message creator before they run Submit Protocol. It ensures only authorized couriers can successfully get data from message creator, and assume only well behaved couriers can get authorized, the probability of success delivery will definitely be increased. Furthermore, if any message is lost by a courier, later it will be very easy to trace back to the troublemaker. Nevertheless, the drawback of it is also non-negligible. Firstly it increases the overhead for key distribution and management as not only end users possess unique key pairs but also every authorized couriers. Secondly it burdens the running of CDSP as mutual authentication is needed when both Submit and Transmit Protocols. As a consequence, in the future implementation, it would be reasonable to provide both two sets of protocols in the application and leave it for users to choose which one to run based on the specific circumstances.

Another imperfection in the protocol design is that: after running the protocol, message originator can not fully deny sending a message to the recipient. As stated in the protocol property specification, this compromise is made to minimize the computational complexity while running the protocol as non-interactive deniable authentication is computational consuming. So, in the future implementation, it is also expected that user can choose which deniable authentication method to use based on the specific circumstances.

Besides, the protocol specification does not cover the corresponding schemes for key distribution, key management and key revocation, instead, they are assumed to be taken care of before running the protocol. Although making such minimum assumptions about the requirements makes the protocol as generic as possible, however, it may reduce the consistency of the whole system. Thus we hope that the best fitted key manipulation schemes can be created for CDSP protocol in the future revised designs.

Also it will be plausible if application can be developed in portable devices, so that tests and evaluations can be done on the portable devices as well. It obviously is more close to the real scenario where only portable devices will be used as couriers in the protocol, and with the extensible and reusable core library developed, it will not be a too time-consuming task.

Finally the efficiency of the protocol always need to be improved as it is one of the main requirement of the protocol design. Efficiency improvement could be accomplished by exploring some more newly invented efficient cryptographic operations and substitute the less efficient ones, or by further revising the code of the implementation.