\section*{Protocol Specification}
A successful run of CDSProtocol consists of 3 stages - (1) message creators submit their messages to a single courier. (2) courier physically transports to the message receiver. (3) courier transmits the messages to the message receiver. As controlling couriers and  planning couriers' routes are out of the scope of this protocol, it is assumed that in this protocol, couriers eventually are able to approach the target. Therefore stage (2) will not be discussed here. \par
The protocol specification will focus on the other two stages - (1) and (3), we call them Message Acquisition phase and Message Delivery phase. To ensure secure communication in both two phases, two sub-protocols - Submit Protocol and Transmit Protocol, are defined for two phases respectively. The Submit Protocol runs between Alice and Courier, while the Transmit Protocol runs between Bob and Courier. As those two sub-protocols can be run independently, they will be explained separately.
\subsection{Notations}
\paragraph{Concatenation \textbar\textbar :}
It denotes the operation that concatenates two pieces of data together. For example ``A\textbar\textbar B" simply means appending B to A.
\paragraph{Repeat concatenation $^+$ :}
It represents a piece of data concatenates itself one or many times. For example ``A$^+$" equals a sequence of A concatenated together, which can be ``A" or ``A\textbar\textbar A" or ``A\textbar\textbar A\textbar\textbar  ..." where A occurs at least once.
\subsection{Submit Protocol}
\subsubsection{Preconditions}
\begin{itemize}
\item Alice holds a unique pair of asymmetric key ($pk_A$, $sk_A$)
\item Courier knows Alice's public key $pk_A$
\end{itemize}
\subsubsection{Postconditions}
\begin{itemize}
\item Alice knows all the messages have been successfully sent to someone
\item Alice doesn't know the identity of the receiver
\item Alice doesn't know whether the message will be eventually deliver to Bob
\item Courier knows the integrity of the message is preserved
\item Courier knows the authenticity of origin of Alice's messages
\end{itemize}

\subsubsection{Message Sequencing Chart}
\begin{msc}{Submit Protocol}
\setlength{\instdist}{3\instdist}
\setlength{\envinstdist}{2\envinstdist}
\setlength{\levelheight}{1.5\levelheight}
\declinst{alice}{Alice}{}
\declinst{courier}{Courier}{}

\action*{pick a random key k$_C$}{courier}
\nextlevel[2]
\mess{1: Courier\textbar\textbar size\textbar\textbar E$_{pk_A}$(k$_C$)}{courier}{alice}
\nextlevel
\action*{pick symmetric key k$_{AB}$}{alice}
\nextlevel
\action*{Meta = E$_{pk_B}$(k$_{AB}$\textbar\textbar Alice\textbar\textbar Bob\textbar\textbar timestamp)}{alice}
\nextlevel
\action*{MetaS = E$_{k_{AB}}$(SIGN$_A$(Meta))}{alice}
\nextlevel
\action*{Msg = E$_{k_{AB}}$(message content)}{alice}
\nextlevel
\action*{MsgM = MAC$_{k_{AB}}$(Msg)}{alice}
\nextlevel
\action*{Infos = Alice\textbar\textbar Bob\textbar\textbar Meta\textbar\textbar MetaS\textbar\textbar Msg\textbar\textbar MsgM}{alice}
\nextlevel[2]
\mess{2: Infos\textbar\textbar MAC$_{k_C}$(Infos)}{alice}{courier}
\nextlevel
\action*{check validity of MAC$_{k_C}$(Infos)}{courier}
\nextlevel
\action*{check validity of sender ID}{courier}
\nextlevel
\action*{check size of Infos}{courier}
\nextlevel[2]
\mess{3: MAC$_{k_C}$(message 2)}{courier}{alice}
\nextlevel
\action*{check validity of MAC$_{k_C}$(message 2)}{alice}
\nextlevel
\end{msc}
\\


\subsubsection{Specification}
\begin{enumerate}
\item Courier randomly creates a symmetric key $k_C$. The key will be sent to Alice and used for deniable authentication.

\item Courier connects to Alice and sends message 1 to Alice. Message 1 should contain (1) The $\mathcal{ID}$ of Courier, (2) The maximum storage size of Courier, (3) Random key $k_C$ encrypted by Alice's public key.

\item Alice randomly creates a symmetric key $ k_{AB} $. The key is supposed to be the session key with Bob.

\item Alice prepares the Meta block, which is an encrypted block under the public key of Bob. The plaintext of Meta contains (1) Symmetric key $ k_{AB} $, (2) The $ \mathcal{ID} $ of Alice, (3) The $ \mathcal{ID} $ of Bob, (4) A timestamp indicates the time Alice creates this message.

\item Alice creates a MetaS block which is an encrypted digital signature of Meta, under the key $ k_{AB} $.

\item Alice creates the Msg block. Alice encrypts the message for Bob under $ k_{AB} $.

\item Alice creates a MsgM block, which is the MAC of Msg, created under the $ k_{AB} $.

\item Alice creates a Infos block, which is the concatenation of (1) The $ \mathcal{ID} $ of Alice, (2) The $ \mathcal{ID} $ of Bob, (3) Meta block, (4) MetaS block, (5) Msg block, (6) MsgM block.

\item Alice prepares message 2. Alice first reveals $ k_C $ by decrypting E$_{pk_A}$(k$_C$) in message 1, using $ sk_A $. Then it creates a MAC of the Infos block under $ k_C $ and appends the $ \mathcal{MAC}_{k_C} $ to the Infos block, forming message 2.

\item Alice examines the size of Infos block and sends out message 2. If the total size of Meta, MetaS, Msg, MsgM blocks exceeds the storage limitation of Courier, it either reduces the size of those blocks and prepares them again, or reports an error and aborts the protocol. If it doesn't, Alice sends the whole message 2 to Courier.

\item Courier checks the validity of message 2. It first verifies $ \mathcal{MAC}_{k_C} $(Infos), if true, then checks the $ \mathcal{ID} $ of Alice. After that, it checks the total size of the Meta, MetaS, Msg and MsgM blocks, making sure it does not exceeds the storage limitation. If any of above checks violate, Courier should report an error and abort the protocol.

\item Courier prepares message 3. Courier uses $ k_C $ to create a MAC of whole message 2 which is received from Alice. 

\item Alice checks the validation of message 3. Alice verifies the received MAC using $ k_C $. If it verifies false, Alice knows its message may not be successfully sent, it means the protocol fails. Otherwise, the protocol successes.
\end{enumerate}

\pagebreak
\subsection{Transmit Protocol}
\subsubsection{Preconditions}
\begin{itemize}
\item Bob holds a unique pair of asymmetric key ($pk_B$, $sk_B$)
\item Courier knows Bob's public key $pk_B$
\end{itemize}
\subsubsection{Postconditions}
\begin{itemize}
\item Bob accepts the message, knowing it is created by Alice
\item Bob doesn't know the identity of the message sender
\item Courier knows Bob has successfully received and accepted the message
\end{itemize}
\subsubsection{Message Sequencing Chart}
\begin{msc}{Transmit Protocol}
\setlength{\instdist}{3\instdist}
\setlength{\envinstdist}{2\envinstdist}
\setlength{\levelheight}{1.5\levelheight}
\declinst{bob}{Bob}{}
\declinst{courier}{Courier}{}

\action*{pick a random key k$_C$}{courier}
\nextlevel
\action*{concatenate all Data blocks for Bob}{courier}
\nextlevel[2]
\mess{1: Courier\textbar\textbar E$_{pk_B}$(k$_C$)\textbar\textbar Data$^+$}{courier}{bob}
\nextlevel
\action*{check validity of all Data blocks}{bob}
\nextlevel[2]
\mess{2: MAC$_{k_C}$(message 1)}{bob}{courier}
\nextlevel
\action*{check validity of MAC$_{k_C}$(message 1)}{courier}
\nextlevel
\end{msc}
\\
where: \\
Data = Meta\textbar\textbar MetaS\textbar\textbar Msg\textbar\textbar MsgM\\
Meta = E$_{pk_B}$(k$_{AB}$\textbar\textbar Alice\textbar\textbar Bob\textbar\textbar timestamp)\\
MetaS = E$_{k_{AB}}$(SIGN$_A$(Meta))\\
Msg = E$_{k_{AB}}$(message content)\\
MsgM = MAC$_{k_{AB}}$(Msg)

\subsubsection{Specification}
\begin{enumerate}
\item Courier randomly creates a symmetric key $k_C$. The key will be sent to Bob and used for deniable authentication.

\item Courier prepares the data for Bob. Courier searches for all early collected Data blocks for Bob, each Data block is the concatenation of (1) Meta block, (2) MetaS block, (3) Msg block, (4) MsgM block. Courier then concatenates all founded Data blocks, forming Data$^+$ block.

\item Courier connects to Bob and sends message 1 to Bob. Message 1 contains (1) The $\mathcal{ID}$ of Courier, (2) Random key $ k_C $ encrypted by Bob's public key, (3) The concatenation of all Data blocks for Bob.

\item Bob receives message 1 and checks the validity of all Data blocks received. Specifically, for checking every Data block, it should do the following steps:
 \begin{enumerate}
 \item use $ sk_A $ to decrypt Meta, and reveal $ k_{AB} $, message creator's ID, message recipient ID and timestamp.
 \item check if message recipient ID is Bob itself and check timestamp to see if this message is expired.
 \item use $ k_{AB} $ to decrypt MetaS and reveal SIGN$_A$(Meta).
 \item verify SIGN$_A$(Meta) using $ pk_A $.
 \item verify MsgM using $ pk_A $.
 \item use $ pk_A $ to decrypt Msg and reveal the message content sent.
 \end{enumerate}
If any violation occurs during the checking or verification in certain Data block, Bob just discards the Data block and continues checking other blocks if any.

\item Bob prepares message 2. Bob first reveals $ k_C $ by decrypting E$_{pk_B}$(k$_C$) in message 1, using $ sk_B $. Then Bob uses $ k_C $ to create a MAC of whole message 1 which is received from Courier. 

\item Courier checks the validation of message 2. Courier verifies the received MAC using $ k_C $. If it verifies false, Courier knows its message may not be successfully sent, it means the protocol fails. Otherwise, the protocol successes.
\end{enumerate}
