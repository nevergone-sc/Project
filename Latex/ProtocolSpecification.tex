\chapter{Protocol Specification}
The protocol introduced for the system is called Courier-Dependent Security Protocol, it defines the interactions between Alice (message creator), Bob(message recipient) and Courier in the system described above when they engage communication. A successful run of CDSP consists of 3 stages - (1) message creators submit their messages to a single courier. (2) courier physically transports to the message receiver. (3) courier transmits the messages to the message receiver. As controlling couriers and  planning couriers' routes are out of the scope of this protocol, it is assumed that in this protocol, couriers eventually are able to approach the target. Therefore stage (2) will not be discussed here.

The protocol specification will focus on the other two stages - (1) and (3), we call them Message Acquisition phase and Message Delivery phase. To ensure secure communication in both two phases, two sub-protocols - Submit Protocol and Transmit Protocol, are defined for those two phases respectively. The Submit Protocol runs between Alice and Courier, it defines how message is submitted from Alice to Courier, while the Transmit Protocol runs between Bob and Courier, it defines how message is transmitted from Courier to Bob. As those two sub-protocols can be run independently, they will be explained separately.

\section{Notations}
The detail entity operations and message sequences of two sub-protocols will be displayed in message sequencing charts separately. Before showing those charts the notations are introduced first.
\paragraph{Encryption Functions $\mathcal{E}_k$ :}
This notation denotes an abstraction of all encryption functions, including both symmetric and asymmetric encryptions. The subscript k denotes the key used for the encryption processes, and it is used for differentiating symmetric and asymmetric encryption, such like if the key is specified as an asymmetric key, it indicates the function is an asymmetric encryption, while symmetric key indicates symmetric encryption.
\paragraph{Message Authentication Code Function $\mathcal{MAC}_k$ :}
It denotes an abstract MAC function, and the subscript k indicates the key used for the function.
\paragraph{Digital Signature Function $\mathcal{SIGN}_E$ :}
It denotes an abstract digital signature function, and the subscript E indicates the entity who creates this signature using a secret key sk$_A$. And it can only be verified under the corresponding public key pk$_B$.
\paragraph{Concatenation \textbar\textbar :}
It denotes the operation that concatenates two pieces of data together. For example ``A\textbar\textbar B" simply means appending B to A.
\paragraph{Accumulation $^+$ :}
It represents many pieces of data which have same format accumulated together. For example ``ID$^+$" equals a sequence of IDs concatenated together, which can be ``ID$_0$" or ``ID$_0$\textbar\textbar ID$_1$" or ``ID$_0$\textbar\textbar ID$_1$\textbar\textbar  ..." where it contains at least one ID and the content of IDs can be different.
\section{Submit Protocol}
\subsection{Preconditions}
\begin{itemize}
\item Alice holds a unique pair of asymmetric key ($pk_A$, $sk_A$)
\item Courier knows Alice's public key $pk_A$
\end{itemize}
\subsection{Postconditions}
\begin{itemize}
\item Alice knows all the messages have been successfully sent to someone
\item Alice doesn't know the identity of the receiver
\item Alice doesn't know whether the message will be eventually deliver to Bob
\item Courier knows the integrity of the message is preserved
\item Courier knows the authenticity of origin of Alice's messages
\end{itemize}

\subsection{Message Sequencing Chart}
\begin{msc}{Submit Protocol}
\setlength{\instdist}{3\instdist}
\setlength{\envinstdist}{2.4\envinstdist}
\setlength{\levelheight}{1.5\levelheight}
\declinst{alice}{Alice}{}
\declinst{courier}{Courier}{}

\action*{pick a random key $k_C$}{courier}
\nextlevel[2]
\mess{1: $ \mathcal{ID}_C $\textbar\textbar $ \mathcal{N}_s $\textbar\textbar $\mathcal{E}_{pk_A}$($k_C$)}{courier}{alice}
\nextlevel
\action*{\parbox{8.7cm}{
1. pick symmetric key $k_{AB}$ \\
2. Meta = $\mathcal{E}_{pk_B}$($k_{AB}$\textbar\textbar $ \mathcal{ID}_A $\textbar\textbar $ \mathcal{ID}_B $\textbar\textbar timestamp)\\
3. MetaS = $\mathcal{E}_{k_{AB}}$($\mathcal{SIGN}_A$(Meta))\\
4. Msg = $\mathcal{E}_{k_{AB}}$(message content)\\
5. MsgM = $\mathcal{MAC}_{k_{AB}}$(Msg)\\
6. Info = $ \mathcal{ID}_A $\textbar\textbar $ \mathcal{ID}_B $\textbar\textbar Meta\textbar\textbar MetaS\textbar\textbar Msg\textbar\textbar MsgM
}}{alice}
\nextlevel[6]
\mess{2: Info\textbar\textbar $\mathcal{MAC}_{k_C}$(Info)}{alice}{courier}
\nextlevel
\action*{\parbox{6.1cm}{
1. check validity of $\mathcal{MAC}_{k_C}$(Info)\\
2. check validity of sender ID\\
3. check size of Info
}}{courier}
\nextlevel[4]
\mess{3: $\mathcal{MAC}_{k_C}$(message 2)}{courier}{alice}
\nextlevel
\action*{check validity of $\mathcal{MAC}_{k_C}$(message 2)}{alice}
\nextlevel
\end{msc}
\\


\subsection{Specification}
Before the start of the protocol, Courier uses a symmetric key generator to create a random symmetric key $k_C$ which will be sent to Alice and used for deniable authentication later. The key $ k_C $ must fit the symmetric encryption / decryption scheme which will be used in  later of the protocol. Then Courier actively connects to Alice and sends first message to Alice. The first message should contain (1) The ID of Courier $\mathcal{ID}_C$, (2) The maximum storage size of Courier $ \mathcal{N}_s $, (3) Symmetric key $k_C$ encrypted by Alice's public key, which is $\mathcal{E}_{pk_A}$($k_C$).

Once Alice receives the first message from Courier, it prepares for sending the second message as reply. Firstly, Alice creates a random symmetric key $ k_{AB} $ which is supposed to be used as the session key with the message recipient Bob. Then Alice prepares the Meta block and its digital signature block MetaS. The Meta block contains  meta information of the message content for Bob and it is encrypted under the public key of Bob so that it is assumed only Bob can access the content of Meta block. The content of Meta block contains (1) Symmetric session key $ k_{AB} $, (2) The ID of message creator (Alice) $ \mathcal{ID}_A $, (3) The ID of message recipient (Bob) $ \mathcal{ID}_B $, (4) A timestamp indicates the exact time when Alice creates this message. To prove the authenticity of the Meta block, Alice creates a digital signature of Meta block using her secret key sk$_A$. However sending raw signature of Meta block release the authenticity of its origin to the Courier, in which case Courier will be able to prove the fact that the Meta block comes from Alice. Thus, to hide the authenticity of the origin of the Meta block, the signature is contained in MetaS block where it is encrypted with $ k_{AB} $ so that only Bob can reveal it. The reason of encrypting signature under symmetric session key rather than Bob's public key is for efficiency concern.

After Alice has Meta and MetaS blocks, it creates a Msg block which is the encrypted message content for Bob under $ k_{AB} $. To ensure the integrity of the Msg block, a MAC of Msg block is created under $ k_{AB} $, the MAC forms MsgM block.

After Alice gets above 4 blocks, it creates a Info block which reflects ``all information" for Courier. Info block is the concatenation of (1) The ID of message creator (Alice) $ \mathcal{ID}_A $, (2) The ID of message recipient (Bob) $ \mathcal{ID}_B $, (3) Meta block, (4) MetaS block, (5) Msg block, (6) MsgM block. Then Alice appends a MAC of the Info block to ensure its integrity. The MAC key is $ k_C $, which can be revealed by decrypting $\mathcal{E}_{pk_A}$($k_C$) in the first received message, and it proves the identity of Alice to the Courier. Finally Alice examines the size of Info block. If the total size of Meta, MetaS, Msg, MsgM blocks exceeds the storage limitation of Courier $ \mathcal{N}_s $, it either reduces the size of those blocks and prepares them again, or reports an error and aborts the protocol. If it doesn't, Alice sends the whole message 2 which contains Info block and its MAC, to Courier.

Upon receipt of the message 2 from Alice, Courier checks the validity of message 2. It first verifies $ \mathcal{MAC}_{k_C} $(Info), if it fails it means either the Info block has been modified or the message sender used a wrong key, both leads to Courier abort the protocol. If it verifies true then Courier checks the ID of message creator to see if it is indeed the entity it is going to communicate (here whether it is $ \mathcal{ID}_A $). After that, it checks the total size of the Meta, MetaS, Msg and MsgM blocks, making sure it does not exceeds the its storage limitation. If any of above checks violate, Courier should report an error and abort the protocol. If all checks success, Courier uses $ k_C $ to create a MAC of the whole message 2 received from Alice and sends it to Alice as a confirmation message.

At the end of the protocol, Alice checks the validation of message 3. It verifies the received MAC using $ k_C $. If it verifies true, it means the protocol success and all postconditions are held. Otherwise, the protocol fails. However, the verification result does not prove the fact whether Courier has received the correct message 2 or not. Alice only knows its message may or may not be successfully sent. So further actions can be taken by Alice such as waiting for next Courier to send the same message or report an error, and they are not restricted by this protocol.


\pagebreak
\section{Transmit Protocol}
\subsection{Preconditions}
\begin{itemize}
\item Bob holds a unique pair of asymmetric key ($pk_B$, $sk_B$)
\item Courier knows Bob's public key $pk_B$
\end{itemize}
\subsection{Postconditions}
\begin{itemize}
\item Bob accepts the message, knowing it is created by Alice
\item Bob doesn't know the identity of the message sender
\item Courier knows Bob has successfully received and accepted the message
\end{itemize}
\subsection{Message Sequencing Chart}
\begin{msc}{Transmit Protocol}
\setlength{\instdist}{3\instdist}
\setlength{\envinstdist}{2\envinstdist}
\setlength{\levelheight}{1.5\levelheight}
\declinst{bob}{Bob}{}
\declinst{courier}{Courier}{}

\action*{\parbox{6.7cm}{
1. pick a random key $k_C$\\
2. accumulate all Data blocks for Bob
}}{courier}
\nextlevel[4]
\mess{1: $ \mathcal{ID}_C $\textbar\textbar $\mathcal{E}_{pk_B}$($k_C$)\textbar\textbar Data$^+$}{courier}{bob}
\nextlevel
\action*{check validity of all Data blocks}{bob}
\nextlevel[2]
\mess{2: $\mathcal{MAC}_{k_C}$(message 1)}{bob}{courier}
\nextlevel
\action*{check validity of $\mathcal{MAC}_{k_C}$(message 1)}{courier}
\nextlevel
\end{msc}
\\
where: \\
Data = Meta\textbar\textbar MetaS\textbar\textbar Msg\textbar\textbar MsgM\\
Meta = $\mathcal{E}_{pk_B}$($k_{AB}$\textbar\textbar $ \mathcal{ID}_A $\textbar\textbar $ \mathcal{ID}_B $\textbar\textbar timestamp)\\
MetaS = $\mathcal{E}_{k_{AB}}$($\mathcal{SIGN}_A$(Meta))\\
Msg = $\mathcal{E}_{k_{AB}}$(message content)\\
MsgM = $\mathcal{MAC}_{k_{AB}}$(Msg)

\subsection{Specification}
Before start of the protocol, Courier first creates a random symmetric key $k_C$ which will be sent to Bob and used for deniable authentication. Once Courier has $ k_C $, it prepares the data for Bob. Assuming by the time Courier starts the Transmit Protocol with Bob, it already has certain pieces of data for Bob which are collected earlier from various message creators. We define a Data block as essential data that is submitted by a message creator and need to be transmitted to a message recipient. Each Data block contains (1) Meta block, (2) MetaS block, (3) Msg block, (4) MsgM block. Courier now fetches all those data, and accumulate them together, forming a Data$^+$ block. Then Courier actively connects to Bob and sends the first message to Bob. Specifically, the message 1 contains (1) The ID of Courier $\mathcal{ID}_C$, (2) Symmetric key $k_C$ encrypted by Bob's public key, which is $\mathcal{E}_{pk_B}$($k_C$), (3) The accumulation of all Data blocks for Bob.

Upon receipt of the first message, Bob checks the validity of all Data blocks received. Basically, Bob will chop the Data$^+$ block up and check each Data block one by one. It will accept all Data blocks that pass all verifications while discard all Data blocks that fail any of the verifications. The detailed algorithm used for checking all Data blocks is displayed below in the algorithm ``check all Data blocks".

According to the checking algorithm, there are 4 different checks for each Data block: First Bob checks the recipient ID contained in its Meta block, to see if this message is for Bob himself. Then Bob checks the timestamp contained in the message Meta block to ensure the message is not expired. The detail implementation of such timestamp checking is not defined in this protocol, it can be implemented in various ways by different applications. After checking the timestamp, Bob checks the digital signature of the whole Meta block of the Data block, by verifying it using the public key of the message creator - whose ID has been given in the Meta block. If the signature verifies true, Bob knows for sure that this message is originated from the message creator it indicates (here Alice). Finally, Bob checks the MAC of Msg block by verifying the MsgM block under the message creator's public key. If the result is true, Bob is convinced that the message content is created by the message creator (Alice).

\begin{algorithm}[H]
 \KwData{The concatenation of all Data blocks}
 \KwResult{Bob accepts valid Data blocks and discards invalid ones}
 initialization\;
 \While{there is Data block unchecked}{
  currentDataBlock = nextDataBlock\;
  use $ sk_A $ to decrypt Meta block in currentDataBlock, reveal \{$ k_{AB} $, creatorID, recipientID, timestamp\}\;
  \eIf{recipientID != ``Bob"} {
   discard currentDataBlock
  }{
   \eIf{timestamp expires}{
    discard currentDataBlock
   }{
    use $ k_{AB} $ to decrypt MetaS in currentDataBlock, reveal $\mathcal{SIGN}_A$(Meta)\;
    verify $\mathcal{SIGN}_A$(Meta) using $ pk_A $\;
    \eIf{verifies false}{
     discard currentDataBlock
    }{
     verify MsgM in currentDataBlock using $ pk_A $\;
     \eIf{verifies false}{
      discard currentDataBlock
     }{
      use $ pk_A $ to decrypt Msg and reveal the message content\;
      accepts the message content
     }
    }
   }
  }
 }
\caption{Check all Data blocks}
\end{algorithm}
\bigskip
\bigskip

After all Data blocks have been received and checked, Bob send back a MAC of whole message 1 received from Courier as message 2. The key used for creating the MAC is $ k_C $ which is revealed by decrypting $\mathcal{E}_{pk_B}$($k_C$) in message 1. The message 2 is the confirmation of Bob receiving the message 1.

Finally Courier checks the validation of message 2. It verifies the received MAC using $ k_C $. Same to the Submit Protocol, if the last MAC is verified true, it means the protocol success and all postconditions are held. Otherwise, the protocol fails. However, the verification result does not prove the fact whether Bob has received the correct message 2 or not. Courier only knows its message may or may not be successfully sent. So further actions can be taken by Courier such as restart the protocol again or report an error, and they are not restricted by this protocol.
