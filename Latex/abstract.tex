\begin{abstract}
On the contrary of modern Internet - which provides reliable and continuous connection between end-to-end users, the so called ``challenged networks" denotes a group of networks which are characterized by high latency, restricted bandwidth limitation, short node longevity or bad path stability. In those challenged networks, typical TCP/IP protocol will not be functioning any more, thus other ways have to be used to establish connections within these networks. One of the common methods to achieve communication within challenged networks appears to be using portable devices as couriers to deliver the message for end users. Although this method has been widely used, it is spotted that this intuitive method leaks the information about communication content thus extra work need to be done to ensure its secure communication.
\\

To solve the problem, this thesis aims to create a new protocol to allow secure and efficient communication established in the courier-dependent networks. In this these, first it formalizes such courier-dependent communications by creating a specific communication scenario which takes its security properties into account. Then it proposes a secure protocol for the specific scenario. The protocol provides mainly 4 security guarantees during the communication - authentication, authenticity of origin, confidentiality and deniability. After that, an application of the protocol is developed and the evaluation of the application proves its efficiency and scalability meets the general requirement as an applicable implementation.
\end{abstract}