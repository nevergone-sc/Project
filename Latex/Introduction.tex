\chapter{Introduction}
\section{Motivation}
In modern world, Internet plays a significant role in everyone's daily life as it breaks the geographical boundary and provides services all over the world. Its major achievement is that it allows each pair of end-to-end entities to build stable and reliable connections between each other so that communication becomes no longer a problem. However, there are still many places out of the reach of Internet. A simple example would be a isolated village in the deep of a mountain where infrastructures have not been built, or some extreme environments like interplanetary or under deep ocean where stable and continuous connectivity can not be achieved. In these cases, Internet may fail to provide its service. Entities in such environments can be abstracted as off-line entities as Internet is no longer available to them.

One way to achieve communication between such off-line entities is using a portable device as a courier to deliver messages. Assuming there are two off-line entities Alice and Bob, and Alice wants to send Bob a message. What could happen is a portable device (maybe several) called Courier copies the message from Alice, physically carries the message to Bob and then delivers the message to Bob. Through this intuitive and practical method, the communication between is achieved.

However, despite the probable efficiency defect it may have, we concern about security issues of this kind of communication. Assuming the content of the message is highly confidential, how to prove the authenticity of the message? How to protect it from being revealed to other entities under the threat that the Courier might be intercepted and the data it carries can be examined during the transportation? Bearing those questions in mind, we construct a specific scenario where these security issues are taken into account:

Assuming there are some off-line entities Alice$_0$, Alice$_1$, ... Alice$_m$, they want to send secret messages to another off-line entity Bob by using portable devices as couriers to help delivering the message. However, these entities are separated by a border - Alice$_0$, Alice$_1$, ... Alice$_m$ are located on one side of the border while Bob is located on the other side of the border, and crossing the border not only is time consuming but also requires all data examined by the security guard. Thus, to make a successful delivery, a Courier first approaches Alice$_0$, Alice$_1$, ... Alice$_m$ one by one and get messages from them. Then it crosses the border carrying those messages and be examined. Finally it approaches Bob and delivers those messages to it. As each message sent by Alice is considered highly confidential, it is expected that the message content will never be leaked to a third party (including Courier).

This project is designated to design and implement a protocol for this specific scenario to allow this kind of courier-dependent connection established efficiently and securely between Alices, Bob and the Courier. The protocol name is called Courier-Dependent Security Protocol, or CDSP as abbreviation. Although CDSP is created under above specific settings, the application of it can be generic, as it can be used in many courier-dependent communication scenarios.

\section{Aims and Objectives}
As stated above, this project aims to create a protocol between off-line entities Alices, Bob and Courier to provide security assurance for the message content they exchanged. The detailed scenario has been briefly introduced and the basic requirements have been clarified. More concretely, the invented CDSP takes following properties into consideration:

\begin{itemize}
\item Authentication \\
As the cost of such communication will be high - the transportation of courier is costly, it requires that the Courier only delivers message for authenticated entity. Namely, if Courier wants to deliver message for Alice$_0$, Alice$_0$ must prove its identity before start submitting its message. Same authentication is required for Bob - Bob has to prove its identity before Courier transmits the message to Bob.

\item Authenticity of Origin \\
The authenticity of origin of messages should be preserved during the communication, which means when message recipient gets the message, he is able to ensure the message creator. Meanwhile, it implies the integrity of message should be preserved.

\item Confidentiality of Message Content \\
The message content carried by courier should be kept confidential to any entities except the message creator and recipient. Because there is possibility that the courier could be compromised and data it carries may be examined by third parties, courier should gain no knowledge of the actual content it carries.

\item Deniability \\
As mentioned above, in secret delivery missions, it would be plausible if Alice is able to deny sending the message in such a way that those who got the message can not prove its authentication to any third parties.

\item Efficiency \\
Due to the potential limitation of computing and storage capability of Courier and the high cost of such communication, protocol should be designed in such a way that it uses least number of messages and smallest message size to achieve the goal. Especially for cryptography operations, where overhead for encrypting/decrypting messages could be high.
\end{itemize}

\noindent
At the end of the project, following objectives must be achieved:
\begin{enumerate}
\item A fully specified Courier-Dependent Security Protocol should be created and it should meet the requirements mentioned in the above list.
\item A Java library should be developed providing essential functions for implementing the protocol.
\item An application should be built to actually run this protocol.
\item A test of the application should be done to evaluate the performance of the protocol application.
\end{enumerate}

\section{Dissertation Structure}
In the rest of the dissertation, the detailed work will be presented. In Chapter 2 some related works will be discussed, and the differences of this project with those works will be highlighted. Then Chapter 3 will thoroughly describe the system that the protocol serves, including all set-ups, assumptions and rules. It is designated to convey a detailed picture of the scenario. Chapter 4 will fully illustrate the specification of CDSP with the help of message sequencing charts and afterwards, in Chapter 5, the security properties of it will be highlighted and verified. After that, the implementation of the protocol application will be demonstrated in Chapter 6. No concrete code but graphs and charts will be given to give a high level understanding of the work. Then it moves to Chapter 7, where the design of tests and evaluations of the application will be shown and some data will be analysed to show the performance of the protocol. Finally, Chapter 8 will summarise the achievements of the project and draw some reasonable conclusions by pointing out the current system limitation and potential improvements in future work.